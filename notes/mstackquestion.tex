%%%%%%%%%%%%%%%%%%%%%%%%%%%%%%%%%%%%%%%%%%%%%%%%%%%%%%%%%%%%%%%%%%%%%%%%%%%%%
%% 
%% Informal notes on k-randomization.
%% 
%%%%%%%%%%%%%%%%%%%%%%%%%%%%%%%%%%%%%%%%%%%%%%%%%%%%%%%%%%%%%%%%%%%%%%%%%%%%%


\documentclass[11pt,draft]{article}
%\documentclass[11pt,draft]{amsart}

% Custom styling.
\usepackage{mozdp}
%% Controls enumeration labels
%\usepackage{enumerate}
%% Shrinks margins 
\usepackage{fullpage}
%% Shows equation label keys
%\usepackage[notref]{showkeys}
%% Title matter
\title{}
\author{Maxim Zhilyaev \and David Zeber}


%%%%%%%%%%%%%%%%%%%%%%%%%%%%%%%%%%%%%%%%%%%%%%%%%%%%%%%%%%%%%%%%%%%%%%%%%%%%%

\begin{document}
\maketitle

\section{MSTACK QUESTION}
Bounding ratio of probabilities of Poisson-Binomial Distribution.

There are $N$ Bernoulli trials, where $m$  trails have probability of success $p$ and $N-m$ trails have probability of success $q=1-p$.  Assume $p>q$.
The number of successes is a random variable $S$ expressed as a sum of two binomial random variables:

\begin{align}
S = Bin(p,m) + Bin(q,N-m)
\end{align}

The distribution of $S$ is known to be a Poisson-Binomial Distribution.
I am studying the behavior of the ratio between P(S=k) and P(S=k-1) with respect to $m$ and $k$.
For a given $m$ and number of successes $k$, denote such ratio as $R(k,m)$:

\begin{align}
R(k,m)  = \frac{P(S=k \; | \;m)}{P(S=k-1 \; |\;m)}
\end{align}

I am particularly interested in the behavior of this ratio for small quantiles, and ran numerical simulation for $R(k,m)$ when $k << mean$. 
It appears that for values of $k$ that are equally distant from the mean, the following holds:

Let $\mu_m=m \cdot (p-q) + N \cdot q$ be the mean of the corresponding distribution and $\alpha$ be the distance away from the mean.
Then for $k=\mu_m - \alpha$:

\begin{align}
R(\mu_0 - \alpha,0)  > R(\mu_1 - \alpha,1) > \dots > R(\mu_{m-1} - \alpha,m-1) > R(\mu_{m} - \alpha,m) > \dots > R(\mu_{N} - \alpha,N) 
\end{align}

The ratio seems to bounded by two binomial distribution for $m=0$ and $m=N$ respectively.

It is easy to see why  $R(\mu_0 - \alpha,0) >  R(\mu_{N} - \alpha,N)$.

\begin{align}
R(k,0) = \frac{\binom{N}{k} q^kp^{N-k}}{\binom{N}{k-1} q^{k-1}p^{N-k+1}} = \frac{N-k+1}{k} \cdot \frac{q}{p} \\
R(k,N) = \frac{\binom{N}{k} q^kp^{N-k}}{\binom{N}{k-1} q^{k-1}p^{N-k+1}} = \frac{N-k+1}{k} \cdot \frac{p}{q} 
\end{align}

Setting  $k=\mu - \alpha$ for each case, we have
\begin{align}
R(k,0) =  \frac{N-qN + \alpha+1}{qN-\alpha} \cdot \frac{q}{p} \approx \frac{qp + \frac{\alpha q}{N}}{qp - \frac{\alpha p}{N}}\\
R(k,N) = \frac{N-pN + \alpha+1}{pN-\alpha} \cdot \frac{p}{q} \approx \frac{qp + \frac{\alpha p}{N}}{qp - \frac{\alpha q}{N}}\\
\frac{qp + \frac{\alpha q}{N}}{qp - \frac{\alpha p}{N}} >  \frac{qp + \frac{\alpha p}{N}}{qp - \frac{\alpha q}{N}} \\
(qp + \frac{\alpha q}{N})(qp - \frac{\alpha q}{N}) > (qp + \frac{\alpha p}{N})(qp - \frac{\alpha p}{N}) \\
(qp)^2 - \left ( \frac{\alpha q}{N} \right )^2 > (qp)^2 - \left ( \frac{\alpha p}{N} \right )^2 
\end{align}

Since $p>q$, the above inequality holds.

But I am unable to express $R(k,m)$ analytically, no prove that it is bounded between ratios corresponding to the limiting binomial distributions (although I see it in simulations).
Any help with proving this and/or pointers to related papers will be much appreciated.


\end{document}

