%%%%%%%%%%%%%%%%%%%%%%%%%%%%%%%%%%%%%%%%%%%%%%%%%%%%%%%%%%%%%%%%%%%%%%%%%%%%%
%% 
%% Informal notes on k-randomization.
%% 
%%%%%%%%%%%%%%%%%%%%%%%%%%%%%%%%%%%%%%%%%%%%%%%%%%%%%%%%%%%%%%%%%%%%%%%%%%%%%


\documentclass[11pt,draft]{article}
%\documentclass[11pt,draft]{amsart}

% Custom styling.
\usepackage{mozdp}
%% Controls enumeration labels
%\usepackage{enumerate}
%% Shrinks margins 
\usepackage{fullpage}
%% Shows equation label keys
%\usepackage[notref]{showkeys}
%% Title matter
\title{}
\author{Maxim Zhilyaev \and David Zeber}


%%%%%%%%%%%%%%%%%%%%%%%%%%%%%%%%%%%%%%%%%%%%%%%%%%%%%%%%%%%%%%%%%%%%%%%%%%%%%

\begin{document}
\maketitle

\section{recursion formular - simple version}

Consider generating function for Poisson-Binomial of you case.  $m$ Bernoulli trials with success $p$ and $n$ Bernoulli trials with success $q$. 

\begin{align}
m+n = N \\
q+p = 1 \\
G(x) = (q+px)^m \cdot (p+qx)^n
\end{align}

The derivative of $ln(a(x)$ is given by:

\begin{align}
[ln(g(x)]' = \frac{g'(x)}{g(x)} = \frac{\left ( \sum_{i=0}^{N} a_i x^i \right )^`}{ \sum_{i=0}^{N} a_i x^i} = \frac{\sum_{i=1}^{N} i \cdot a_i x^{i-1}}{  \sum_{i=0}^{N} a_i x^i}
\end{align}

On the other hand
\begin{align}
[ln(g(x))]' = (m(q+px) + n(p+qx))^` = \frac{mp}{q+px} + \frac{nq}{p+qx} = \frac{xpq(m+n) + mp^2 + nq^2}{x^2pq + x(p^2 + q^2) + pq}
\end{align}

Equating both expressions we get
\begin{align}
\frac{xpqN + mp^2 + nq^2}{x^2pq + x(p^2 + q^2) + pq} = \frac{\sum_{i=1}^{N} i \cdot a_i x^{i-1}}{  \sum_{i=0}^{N} a_i x^i} \\
(xpqN + mp^2 + nq^2)(\sum_{i=0}^{N} a_i x^i) = (x^2pq + x(p^2 + q^2) + pq)(\sum_{i=1}^{N} i \cdot a_i x^{i-1})
\end{align}

Multiplying and equating terms with same power of $x$ we get:
\begin{align}
a_i(mp^2 + nq^2) + a_{i-1}pqN = a_{i+1}pq(i+1) + a_i(p^2 + q^2)i + a_{i-1}pq(i-1) \\
a_{i-1}pq(N - i +1) = a_{i+1}pq(i+1) + a_i[(p^2 + q^2)i - (mp^2 + nq^2)] \\
a_{i-1}(N - i +1) = a_{i+1}(i+1) + a_i\frac{(p^2 + q^2)i - (mp^2 + nq^2)}{pq}
\end{align}

Denote the expectation of successes as $\mu$, and note the obvious relationships:
\begin{align}
mp^2 + nq^2 = mp^2 + (N-m)q^2 = m(p^2 - q^2) + Nq^2 =\\
m(p- q) +Nq -Nq +Nq^2 = \mu + Nq(q-1) = \mu - Npq\\
p^2 - q^2 = (p+q)^2 - 2pq = 1 - 2pq
\end{align}

Replacing the expression above in 1.10, we have:
\begin{align}
a_{i-1}(N - i +1) = a_{i+1}(i+1) + a_i\frac{(1 -2pq)i - (\mu - Npq)}{pq} \\
a_{i-1}(N - i +1) = a_{i+1}(i+1) + a_i \left (N - 2i - \frac{\mu - i}{pq} \right ) \\
\frac{a_{i-1}}{a_{i}} (N - i +1) = \frac{ a_{i+1}}{a_i} (i+1) + N - 2i  - \frac{\mu - i}{pq} \\
\frac{a_{i-1}}{a_{i}} (N - i +1) = \frac{ a_{i+1}}{a_i} (i+1) + N - i +1 - i - 1 - \frac{\mu - i}{pq} \\
\frac{a_{i-1}}{a_{i}} (N - i +1) = N - i +1 - (\frac{ a_{i+1}}{a_i} - 1)(i+1) - \frac{\mu - i}{pq}
\end{align}

Rearranging the terms, we arrive to the recursive relationship between probabilistic ratios at $i$ and $i-1$:
\begin{align}
\frac{   N - i + 1}{ N - i +1  + (\frac{ a_{i+1}}{a_i} - 1)(i+1)   - \frac{\mu - i}{pq}   } =  \frac{a_{i}}{a_{i-1}} 
\end{align}

\textbf{NOTATION:}

Denote the distance between $\mu$ and $i$ as $l$.  Also denote the probabilistic ratio at $i$ as $f_i$.  Then:
\begin{align}
\mu - i = l\\
i = \mu -l \\
f_{i+1} = f_{l-1} = \frac{a_{i+1}}{a_{i}} \\
f_{i} =  f_{l} = \frac{a_{i}}{a_{i-1}} \\
f_{i-1} =  f_{l+1} = \frac{a_{i-1}}{a_{i-2}} \\
\frac{   N - i + 1}{ N - i + 1  + (f_{l-1} - 1)(i+1)  - \frac{l}{pq}   } =  \frac{  1 }{ 1  - \frac{1}{N- i + 1}\left (\frac{l}{pq} - (f_{l-1} - 1)(i+1) \right ) }  = f_{l}  \\
\frac{   N - i + 2}{ N - i + 2  + (f_l - 1)i  - \frac{l+1}{pq}   } =   \frac{  1 }{ 1  - \frac{1}{N- i + 2}\left (\frac{l+1}{pq} - (f_{l} - 1)i \right ) }  = f_{l+1} 
\end{align}


For some value of $m$ denote corresponding expectation as $\mu_m$.  Denote the number of successes equal as $i_m$, and a corresponding probability ratio at $i_m$ as $f_{m,l}$:
\begin{align}
\mu_m = mp + (N-m)q\\ 
i_m = \mu_m - l \\
f_{m,l} = \frac{a_{{i_m+1}}}{a_{i_m}}
\end{align}

Expressing $i_m=\mu_m-l$ , allows to re-write formula $1.26$  as follows:

\begin{align}
\frac{  1 }{ 1  - \frac{1}{N- \mu_m + l + 2}\left (\frac{l+1}{pq} - (f_{m,l} - 1)(\mu_m-l) \right ) }  = f_{m,l+1} 
\end{align}


\textbf{Properties of $f_{0,l}$}

Consider probability ratio for the case $m=0$.  Since all Bernoulli trials generate successes with probability $q$, the following relationships hold:
\begin{align}
\mu_0 = Nq \\
f_{0,l} = \frac{a_{i}}{a_{i-1}} = \frac{\binom{N}{i} q^{i}p^{N-i }}{\binom{N}{i-1} q^{i-1}p^{N-i+1}} = \frac{N-i+1}{i}\frac{q}{p} 
\end{align}

Subtracting 1 from both sides of the equality above gives:
\begin{align}
f_{0,l} -1 =   \frac{N-i+1}{i}\frac{q}{p}  - 1 =  = \frac{qN - qi + q - pi}{pi} = \frac{\mu_0 - i + q}{pi} = \frac{l + q}{p(\mu_0 - l)} \\
f_{0,l} -1 =   \frac{l + q}{p(\mu_0 - l)} =  \frac{l + q}{pi}
\end{align}


\textbf{THEOREM 1}
 
For $l > \sigma$ probability ratio $f_{0,l+1}$ at $l+1$ is higher then probability ratio $f_{m,l}$ for any $m$.

\textbf{PROOF:}

Suppose that there's a particular value of $m$ such that $f_{m,{l+1}}  > f_{0,{l+1}}$  for some $l$ and consider $f_{m,{l+1}}$ and $f_{0,{l+1}}$ for $m=m$ and $m=0$:
\begin{align}
\frac{  1 }{ 1  - \frac{1}{N- \mu_m + l + 2}\left (\frac{l+1}{pq} - (f_{m,l} - 1)(\mu_m-l) \right ) }  = f_{m,l+1} \\
\frac{  1 }{ 1  - \frac{1}{N- \mu_0 + l + 2}\left (\frac{l+1}{pq} - (f_{0,l} - 1)(\mu_0-l) \right ) }  = f_{0,l+1} 
\end{align}

For  $f_{m,{l+1}}  > f_{0,{l+1}}$  to hold the difference of denominators in the expressions for  $f_{0,l+1}$  and  $f_{0,m+1}$ must be positive, hence:

\begin{align}
\frac{1}{N-\mu_m + l+2} \left (\frac{l+1}{pq} - (f_{m,l} - 1)(\mu_m - l) \right ) -  \frac{1}{N-\mu_0 + l+2} \left (\frac{l+1}{pq} - (f_{0,l} - 1)(\mu_0 - l) \right) > 0\\
\frac{l+1}{pq} (\frac{1}{N-\mu_m + l + 2}  - \frac{1}{N-\mu_0 + l + 2} ) + \frac{(f_{0,l} - 1)(\mu_0 - l)}{N-\mu_0 + l+2} - \frac{(f_{m,l} - 1)(\mu_m - l)}{N-\mu_m + l+2} > 0 \\
 \frac{l+1}{pq} \frac{\mu_m - \mu_0}{(N-\mu_m + l+2)(N-\mu_0 + l+2)} + \frac{(f_{0,l} - 1)(\mu_0 - l)}{N-\mu_0 + l+2} - \frac{(f_{m,l} - 1)(\mu_m - l)}{N-\mu_m + l+2} > 0
\end{align}

Suppose $f_{m,l} \ge f_{0,l}$, then we can replace $f_{m,l} - 1$ with $f_{0,l} - 1$, and the inequality should still hold since we subtract a lesser value.

\begin{align}
\frac{l+1}{pq} \frac{\mu_m - \mu_0}{(N-\mu_m + l)(N-\mu_0 + l)}  + \frac{(f_{0,l} - 1)(\mu_m - l)}{N-\mu_m + l}  - \frac{(f_{0,l} - 1)(\mu_0 - l)}{N-\mu_0 + l} > 0\\
\frac{l+1}{pq} \frac{\mu_m - \mu_0}{(N-\mu_m + l+2)(N-\mu_0 + l+2)}  + (f_{0,l} - 1) \left [ \frac{\mu_0 - l}{N-\mu_0 + l+2}  - \frac{\mu_m - l}{N-\mu_m + l+2} \right ]  > 0
\end{align}

The expression inside the brackets simplifies to:
\begin{align}
\frac{\mu_0 - l}{N-\mu_0 + l+2}  - \frac{\mu_m - l}{N-\mu_m + l+2}  = - \frac{(N+2)(\mu_m-\mu_0)}{(N-\mu_m + l)(N-\mu_0 + l)}
\end{align}

From here:
\begin{align}
\frac{l+1}{pq} - (f_{0,l} - 1)(N+2) > 0 \\
\frac{l+1}{pq} > (f_{0,l} - 1)(N+2)
\end{align}

From properties of $f_{0,l}$ we have:
\begin{align}
\frac{l+1}{pq} > (f_{0,l} - 1)(N+2) \\
\frac{l+1}{pq} > \frac{l + q}{p(\mu_0 - l)}(N+2) >   \frac{l + q}{p(\mu_0 - l)}N \\
(l+1)(\mu_0 - l) > Nq(l+q) \\
(l+1)(\mu_0 - l) > \mu_0(l+q)
\end{align}


After simplification we have:
\begin{align}
-l^2 - l +Npq > 0
\end{align}

Since the standard deviation is $\sigma = \sqrt{Npq}$, this condition clearly does not hold for $l>\sigma$.  Which proves an important point. When $l$ exceeds $\sigma$,  $f_{m,l+1}$ could only become greater than $f_{0,l+1}$ if at the previous iteration $f_{m,l}$ is less than $f_{0,l}$.  And conversely, if $f_{m,l} > f_{0,l}$ at $l$, the next iteration $f_{m,l+1} < f_{0,l+1}$.  We now can prove the main statement of the theorem.

Choose a particular value for $l$ such that $l>\sigma$.  For those values of $m$ such that $f_{0,l} > f_{m,l}$ the theorem holds already, since $f_{0,l+1} > f_{0,l}$.  
Now consider those $m$ for which $f_{0,l} < f_{m,l}$ at $l$.  In the next iteration the probability ratio $f_{m,l+1}$ for such $m$ has to be become less than $f_{0,l+1}$.  Hence:
\[ f_{0,l+1} > f_{m+1,l} > f_{m,l} \]

Which proves the theorem.  

\section{Derivative magic}

Taking deirvate by $\mu$ in formula 1.30, we have

\[ F'(\mu) = -\frac{ (f_{m,l}-1)(N+2)pq - (l+1) }{[ pq(N+2 + (f_{m,l}-2)(\mu_m -l )) - (l+1)]^2}  \]

The denominator is always positive, hence the derivate is always negative when:
\[ (f-1)(N+2)pq > (l+1)  \]

Which the exact condition for oscillating ratios (see formula 1.45).  Why?


\end{document}



